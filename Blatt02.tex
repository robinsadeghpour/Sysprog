\documentclass[a4paper,12pt]{article}
\usepackage{listings}
\usepackage {tikz}
\usetikzlibrary {positioning}
\definecolor {processblue}{cmyk}{0.96,0,0,0}

\title{Theorie Blatt02}
\author{Jonas Kuß, Anton Jürß, Daniel Nikulin, Robin Sadeghpour}

\begin{document}
\noindent
    \section*{Aufgabe 2.3}
    \subsection*{a)}
    Prozesse representieren sequentielle Aktivitäten innerhalb eines Systems.\\
    Sie sind dynamische Objekte. Ein Prozess ist eine Art virtueller Rechner für
    eine bestimmte Ausführung eines Programms.\\
    Ein Prozess wird definiert durch den Besitz von Ressourcen, 
    durch eine \\Arbeitsvorschrift und durch einen Aktivitätsträger, welcher die\\
    Verarbeitungsvorschrift ausführt. 
    Da Prozesse schaffen Struktur und Ordnung innerhalb eines \\
    Programms (Prozesshirarchie)
    und ermöglichen neben sequentieller Arbeit, auch parallele (Nebenläufigkeit und Parallelität). 
    Somit ermöglichen Prozesse eine effizientere Nutzung von Ressourcen.
    Prozesse werden von einem laufendem Programm erzeugt und erhalten von diesem Eingaben,
    welche sie verarbeiten und liefer demzufolge Ausgaben. Sie sind somit einem Programm
    eindeutig zugewiesen und können somit als eine Instanz dessen aufgefasst werden.
    \\
    \\
    Quellen: Vorlesungsfolien Kapitel 2, Folien 1-14
    \subsection*{b)}
    \paragraph{Nebenläufigkeit}
    ist eine "logisch simultane Verarbeitung von Operationsströmen" (VL Kapitel 2 Folie 12),
    wobei die Prozesse nicht tatsächlich simultan, bzw. parallel Ablaufen sondern verzahnt auf 
    einem Einprozessorsystem.\\
    Dabei werden mehrere Prozesse mindestens einem Prozessor zugeordnet.
    \paragraph{Parallelität} 
    ist eine tatsächlich simultane Ausführung von Prozessen.\\
    Es werden mehrere Prozesse auf mindestens
    zwei Prozessoren zugeordnet. Somit ist die Parallelität eine Teilmenge der Nebenläufigkeit.
    Zudem sind sind mehrfache Verarbeitungselemente notwendig (z.B. Prozessoren).
    \\
    \\
    Quellen: Vorlesungsfolien Kapitel 2, Folien 12-14
    \subsection*{c)}
    Prozesse können im Betriebssystem dank der Datenstruktur des Process Control Block (PCB)
    implementiert werden. Der PCB ist ein "verwaltungstechnischer Repräsentant des Prozesses" (VL Kapitel 2 Folie 16).
    Der PCB enthält Information über den Prozess, unter anderem die Prozessnummer, oder die Zustandsvariable. \\
    Die Zustandsvariable bezeichnet in welchen Zustand der Prozessor sich befindet (Bereit, Laufend, etc.).\\
    Falls ein Zustandswechsel statt findet, werden jenachdem die aktuellen Registerinhalte in dem PCB abgelegt
    oder von dem PCB geladen, der virtuelle Adressraum innerhalb des PCB umgeschaltet und der Prozesszustand aktualisiert.
    \\
    \\
    Quellen: Vorlesungsfolien Kapitel 2, Folien 16-18
    \subsection*{d)}
    TODO
    \section*{Aufgabe 2.4}
    \subsection*{a)}
    \begin {center}
        \begin {tikzpicture} [
            roundnode/.style={circle, draw=black, fill=white, very thick, minimum size=10mm}
        ]
        %Nodes
        \node[roundnode](a) at (0,0)    {a};
        \node[roundnode](b) at (0,-2)   {b};
        \node[roundnode](c) at (0,-4)   {c};
        \node[roundnode](d) at (3,-1)   {d};
        \node[roundnode](e) at (6,-2)   {e};
        \node[roundnode](f) at (9,0)    {f};
        \node[roundnode](j) at (9,-2)   {j};
        \node[roundnode](g) at (9,-4)   {g};
        \node[roundnode](h) at (12,0)   {h};
        \node[roundnode](i) at (12,-4)  {i};
        \node[roundnode](k) at (15,-2)  {k};

        %Lines
        \draw[->] (a) -- (d);
        \draw[->] (b) -- (d);
        \draw[->] (c) -- (e);
        \draw[->] (d) -- (e);
        \draw[->] (e) -- (f);
        \draw[->] (e) -- (j);
        \draw[->] (e) -- (g);
        \draw[->] (f) -- (h);
        \draw[->] (e) -- (f);
        \draw[->] (j) -- (k);
        \draw[->] (g) -- (i);
        \draw[->] (i) -- (k);
        \end{tikzpicture}
    \end{center}
    \subsection*{b)}
    \begin{lstlisting}
        a
        fork b
        fork c
        join b 
        d
        join c
        e 
        fork PartOne
        fork PartTwo
        j
        join PartTwo
        k


        PartOne:                PartTwo:
        f                       g
        h                       j
        end                     end
    \end{lstlisting}
    \indent
\section*{Aufgabe 2.5}
\begin{minipage}{.5\textwidth}
    \subsection*{a) fork/join}
    \begin{lstlisting}
        C
        fork Part
        fork B
        fork K
        join B
        D
        E
        fork F
        join K
        G
        join F
        fork J
        Join Part
        I
        join J
        X
    

        Part:
        A
        H
        end
    \end{lstlisting}
\end{minipage}% This must go next to `\end{minipage}`
\begin{minipage}{.5\textwidth}
    \subsection*{a) parbegin/parend}
    \begin{lstlisting}
        parbegin
            begin
                A
                H
            end
            begin
                parbegin
                    B
                    C
                parend
                D
                parbegin
                    E
                    K
                parend
                parbegin
                    F
                    G
                parend
            end
        parend
        parbegin
            I
            J
        parend
        X
    \end{lstlisting}
\end{minipage}





\end{document}
